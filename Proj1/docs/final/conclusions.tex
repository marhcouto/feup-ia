\begin{frame}{Conclusions}
    
    From the different approaches analyzed we were able to draw some conclusions:
    \begin{itemize}
        \item Despite our heuristic functions not being, at our eyes, closely related to the problem and meaningful, they proved to greatly improve our algorithms' performance.
        \item The results show that our admissible heuristics made the \textbf{A* algorithm} find the 'ideal solution', if that would be the solution closest to the root node. However, the non admissible heuristics improved its speed greatly.
        \item In our interpretation of the problem, any valid solution is acceptable and the best solution is the one simply achieved the fastest. In this case, the \textbf{Greedy algorithm} was the best, as it yielded a solution in the least time and iterations.
        \item \textbf{IDS} was a particularly bad approach, as the optimal was not necessary.
        \item \textbf{DFS} as shown the same number of iterations as depth, as, without any limits, because a solution can always be reached through a certain branch, it will not backtrack ever, being effectively a brute-force approach.
    \end{itemize}

\end{frame}

\begin{frame}{Conclusions}
    The project was a success as we were able to
    \begin{itemize}
        \item create a representation of the problem as a search problem
        \item build algorithms capable of solving it and heuristics to aid them
        \item build an environment to test and showcase the algorithms 
        \item analyze the results and understand the differences 
    \end{itemize}
    Implementing the algorithms necessary to solve this problem  has allowed us to deepen our knowledge and experience on search problems and algorithms.
    
\end{frame}