
\begin{frame}{Problem Specification}
    \textbf{Credit card fraud} is a crime easily stopable if detected. However, this detection must be quick to occur, as transactions are supposed to be fast. The \textbf{objective} of this project is to \textbf{develop a machine learning model} based on supervised learning classification algorithms able to distinguish, based on some input data, if a given transaction related to a bank account is fraudulent or not. 
        \vspace{0.5em}
        
    In order to come up with the ideal model, we must experiment with at least three different classification algorithms, as well as carry out an \textbf{Exploratory Data Analasys}.

\end{frame}


\begin{frame}{Problem Specification}

    The \href{https://www.kaggle.com/datasets/dhanushnarayananr/credit-card-fraud}{dataset} contains the following data:
    
    \vspace{0.5}
    
    \begin{itemize}
        \item \textbf{Distance from home} - continuous
        \item \textbf{Distance from last transaction} - continuous
        \item \textbf{Ratio to median purchase price: } Ratio of the value of the transaction to the median transaction value - continuous
        \item \textbf{Repeat retailer:} Whether the retailer where the transaction happened had other transactions registered for the same person
        \item \textbf{Used chip:} Transcation via credit card - discrete, binary
        \item \textbf{Used pin number} - discrete, binary
        \item \textbf{Online order} - discrete, binary
        \item \textbf{Fraud} - discrete, binary - \textbf{label}
    \end{itemize}
    We considered that all distances are expressed in kilometers.
\end{frame}